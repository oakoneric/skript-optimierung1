In diesem Kapitel befassen wir uns mit Techniken zur Lösung ganzzahliger Optimierungsaufgaben. Dabei dürfen einige oder gar alle Variablen diskrete Werte annehmen.
Somit entfällt eine Argumentation über Ableitung, zulässige Richtungen etc. Ganzzahlige Optimierungsaufgaben sind also \enquote{schwieriger} als die zugehörige stetige Relaxation. Dennoch kann, in einigen Fällen, das Lösen diskreter Aufgaben auch zur effizienten Lösung stetiger Aufgaben beitragen, wie folgendes Beispiel verdeutlicht.